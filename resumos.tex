
% resumo em português
\linespread{1}
\setlength{\absparsep}{12pt} % ajusta o espaçamento dos parágrafos do resumo
\begin{resumo}
 Apresentamos aqui um projeto de dissertação a ser submetido ao Programa de Pós Graduação em Educação: Conhecimento e Inclusão Social da Faculdade de Educação da Universidade Federal de Minas Gerais (FaE/UFMG). Buscamos, no decorrer da pesquisa, interpelar a Base Nacional Comum Curricular(BNCC), problematizando-a e desconstruindo-a a partir do Hip-Hop belo-horizontino, em especial os e as seguintes artistas: Licon, nabru, Djonga, Laura Sette, FBC e Iza Sabino. É do nosso interesse questionar o discurso disciplinar que se faz presente na BNCC. Para tanto pretendemos elaborar uma Coletânea Indisciplinar do Rap de BH que reunirá as produções musicais citadas em uma planilha pública que nos servirá, junto com a BNCC, como arquivos culturais de partida para desenvolvermos esta pesquisa. Nos pautaremos em uma atitude terapêutico genealógica encontrando inspiração nas formas de desenvolver investigações de Ludwig Wittgenstein e Michel Foucault, nos direcionando pelo seguinte objetivo de pesquisa: problematizar e desconstruir, ancorados na produção musical dos últimos 3 anos de 6 {\em rappers} belo-horizontinas(os), o discurso disciplinar integrado na BNCC. Esperançamos que essa pesquisa seja um convite para a luta.

 \textbf{Palavras-chave}: Hip-Hop. Educação Matemática. Currículo.
\end{resumo}

% resumo em inglês
%\begin{resumo}[Abstract]
% \begin{otherlanguage*}{english}
%   This is the english abstract.

%   \vspace{\onelineskip}
% 
%   \noindent 
%   \textbf{Keywords}: Palavras-chave. Palavras-chave. Palavras-chave.
% \end{otherlanguage*}
%\end{resumo}

% resumo em francês 
%\begin{resumo}[Résumé]
% \begin{otherlanguage*}{french}
%    Il s'agit d'un résumé en français.
 
%   \textbf{Mots-clés}: Palavras-chave. Palavras-chave. Palavras-chave.
% \end{otherlanguage*}
%\end{resumo}

% resumo em espanhol
%\begin{resumo}[Resumen]
% \begin{otherlanguage*}{spanish}
%   Este es el resumen en español.
  
% \textbf{Palabras clave}: Palavras-chave. Palavras-chave. Palavras-chave.
% \end{otherlanguage*}
% \end{resumo}
